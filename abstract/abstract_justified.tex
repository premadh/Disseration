\documentclass[10pt]{article}

\usepackage[english]{babel}
\usepackage{fouriernc}
\usepackage[T1]{fontenc}
\usepackage[paperwidth=126mm, paperheight=123mm, top=0mm, bottom=0mm, left=0mm, right=0.3mm]{geometry}

\begin{document}

\setlength{\parindent}{3mm}
\small

\noindent When the measurements from the ever improving measurement technologies are accumulated over a period of time, the result is a collection of data in different representations. However, most machine learning and data mining algorithms, in their standard form, are designed to operate on data in a single representation only.


This thesis proposes machine learning and data mining algorithms to analyse data in different representations with respect to resolution within a single analysis. The novel algorithms proposed to analyse multiresolution data are in the field of probabilistic modelling and semantic data mining. First, different deterministic data transformation methods are proposed to transform data across different resolutions. After the data transformation, the resulting datasets in same resolution are integrated and modelled using mixture models.

Second, similar mixture components in a mixture model are merged one by one repetitively to generate a chain of mixture models. A new fast approximation of the Kullback Leibler divergence is derived to determine the similarity of the mixture components. The chain of generated mixture models are useful for comparison purposes, for example, in model selection. Third, mixture components in different resolutions are iteratively merged to model multiresolution data generating models in each modelled resolution that incorporate information from data in other resolutions.

Fourth, a single multiresolution mixture model with multiresolution mixture components is proposed whose mixture components independently have the capabilities of a Bayesian network. Finally, a three part methodology consisting of clustering using mixture models, rule learning using semantic subgroup discovery, and pattern visualisation using banded matrices is developed for comprehensive analysis of multiresolution data.

The multiresolution data analysis methods presented in this thesis improve the performance of the methods in comparison with their single resolution counterparts. Furthermore, the developed methods aim to make the results understandable to the domain experts. Therefore, the developed methods are useful additions in the analysis of chromosomal aberration patterns and the cancer research in general.

\end{document}