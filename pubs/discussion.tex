\chapter[Discussion]{Discussion}
\markboth{Discussion}{}
\label{ch:discussion}

\begin{fquote}[Abigail Adams]Learning is not attained by chance, 
it must be sought for with ardor and attended to with diligence.
\fqsource{Letter to John Quincy Adams (1780)} \end{fquote} 

\begin{synopsis}
The work in the thesis focused on the analysis of multiresolution 
0--1 data. The application area of choice was chromosomal 
aberrations patterns in cancer genomics defined in multiple 
resolutions. The proposed algorithms, mixture models and 
semantic data mining, for analysis of multiresolution data are 
experimented on the chromosomal aberrations data with plausible
results. Furthermore, an efficient method to train a chain of 
mixture models was proposed to aid model selection
in mixture models. Multiresolution modelling methods, and model
selection in mixture models are discussed in this chapter along
with their applicability, limitations, and possible future 
directions of work. The future directions of work discussed
in this chapter concerns specific methods discussed
and developed in the thesis. The future work section in 
Chapter~\ref{ch:summary}, i.e. Section~\ref{s:finfuture},
discusses the overall future work in the multiresolution 
modelling domain. 
\end{synopsis}



\section{Model Selection in Mixture Models}
\label{ss:dmodelselection}
Model selection is an age--old problem in statistics and machine 
learning~\cite{ando2010bayesian,lahiri2001model}. In~\citepub{j1}, 
we do not propose a new model selection criteria but a computationally 
efficient method to train a series mixture models differing only 
in the number of components. The proposed method provides 
additional facilities of computational efficiency, and similarity 
of the mixture models in the chain except for the number of 
components. Therefore, the method is suitable for comparison in 
model selection. The experiments performed on the three datasets
provide evidence of its efficiency and suitability in model 
selection. Furthermore, the proposed mixture model for Bernoulli
distributions can be seamlessly extended to other distributions
such as the Gaussian distribution. 

The proposed method is sensitive to local optima while  learning 
mixture model via EM algorithm~\cite{McLachlan2008emext,wu1983}.
We try to address the challenges of local optima by training 
multiple mixture models once for largest number of mixture 
components before merging the similar components. The best 
mixture model among the trained mixture models is selected to 
calculate the KL divergence among mixture components. The most 
similar components, i.e., the the pair of components with the 
minimum KL divergence are progressively merged to generate a 
chain of mixture models. However, this does not guarantee that the 
EM algorithm reaches global optimum. Avoidance of local 
minima is still an open research problem in optimisation and
also the EM algorithm. Nevertheless, effectiveness, efficiency, 
and seamless scalability of the proposed method makes the 
proposed method, the method of choice for training mixture 
models for model selection.



\section{Multiresolution Analysis and Modelling of 0--1 Data}
\label{s:dmultires}

Algorithms and methods to study and analyze multiresolution
data forms the crux of the thesis. The proposed algorithms 
complement each other and specific algorithm fulfills the 
requirements of a specific application. Nevertheless, 
ample possibilities and challenges for future improvements 
identified in the proposed algorithms and methods are discussed
in the subsequent paragraphs.

\subsection{Data Transformation for Multiresolution Analysis}
\label{ss:dtatr4multiresmdl}

The data transformation methods deterministically transform 
the data across different resolutions in such a way that 
data in different resolutions can be integrated in a single 
resolution. The integrated data in single resolution can 
then be analyzed using a method of choice because the data 
is of the same dimensionality. In~\citepub{c1}, we experiment
with mixture models and pattern mining algorithms generating 
credible results for multiresolution chromosomal aberrations
data.

The data transformation methods are suitable for analysis requiring
high processing speed and robustness. One of such application 
area is stream data mining~\cite{aggarwal2007data,gaber2005,gama2014,indre2014} 
where the requirements are efficient processing and  robustness
in analysis against minor changes occurring in the data. 
Data transformation methods are efficient because
their computation is simple and are robust against small 
changes and outliers; for the data transformation methods
are deterministic given the structure of the multiresolution 
phenomena. Furthermore, data transformation methods are suitable
for applications requiring single resolution models for 
multiresolution data. In hindsight, the data transformation
methods lack probabilistic interpretation. Adding stochasticity in 
those methods is a possible future work, for example,
with foundations on Potts model~\cite{baxter08,potts52}.

\subsection{Merging of Mixture Components}
\label{ss:dmergingmixcompo}

In~\citepub{c2}, we model multiresolution data by generating mixture 
models in each resolution separately in such a way that the models 
in each resolution incorporate the information from other data 
resolutions. The experiments with  chromosomal aberrations data show 
multiresolution mixture models incorporating the interactions 
between data resolutions produce better results compared to the 
individually trained single resolution models. 
The method is suitable for application areas that require models in 
each level of processing resolution such as image processing, 
and computer vision~\cite{sonka2008}. Furthermore, experiments 
in~\citepub{c2} have  shown that merging of mixture components 
also helps in avoiding local optima when experimented on the two 
single resolution models.

Merging of mixture components from different mixture models aids
in modelling interaction among the mixture models in different 
resolutions. An approximation of symmetric KL divergence is used
to compare the similarity of the components in the mixture 
model. The similar components are then merged. However, the
convergence analysis of KL divergence is not studied in detail
in~\citepub{c2}. Furthermore, upsampling and downsampling of the 
parameters of the mixture model adds another complexity to the 
methodology. Additionally, improvements of the multiresolution 
mixture model and avoidance of local optima in single resolution
mixture model have been verified only by the empirical experiments. 
However, solid mathematical foundations and the proofs for the 
improvement are missing. One direction of future work could focus
on mathematical proofs for the empirical evidence in merging 
components for multiresolution modelling.

\subsection{Multiresolution Mixture Components}
\label{ss:dmultiresmdl}

In~\citepub{c3}, a single multiresolution mixture model with multiresolution 
mixture components are proposed and experimented using multiresolution
chromosomal aberrations dataset. Only a single multiresolution model is generated 
in~\citepub{c3}, which is unlike~\citepub{c2} where a model is generated for 
each data resolution. The individual mixture components provide the functionality
of Bayesian networks. The proposed model is suitable for the situations 
requiring generative modelling prowess of probabilistic models. In~\citepub{c3} 
generative property of the Bayesian network helps imputing the missing 
resolutions of the data. Furthermore, the proposed multiresolution mixture
model could be applicable in any domain where the network structure in 
the multiresolution data is consistent across the dimensionality, for 
example, in the image processing domain.


The mixture components used as a Bayesian network model the 
dependency among the nodes in the network. In addition each 
node requires at least two probability values describing the 
probability of the node given the probability of its parent 
node~\cite{barberBRML12}. In contrast, the EM algorithm
assumes IID distributions for the 
samples~\cite{McLachlan2008emext}. Additionally, the EM 
algorithm provides only a single probability value for a 
node, i.e.,  probability of a random variable ($\theta$) 
taking the value 1; two if you consider 1 - the given 
probability ($1-\theta$). Hence, we transform the nodes
to a vector representation to learn the mixture models via 
the EM algorithm. For this reason, future work in 
multiresolution mixture modelling could be to develop EM 
algorithm to directly learn the parameters of mixture 
models when the components are not vectors but a network. 
Furthermore, transformation network representation in 
multiresolution mixture model to vectors and then learning 
the mixture models using the EM algorithm requires structural 
similarity of networks used as the different mixture components. 
Therefore, the future work could focus on relaxing 
this requirement.

\subsection{Multiresolution Analysis by Semantic Data Mining}
\label{ss:dsemantic}

In \citepub{j2}, we propose a three part methodology for comprehensive
analysis of the multiresolution data. We use clustering results from the
mixture models as the labels for the semantic data mining algorithm. 
The additional background knowledge consists of taxonomy of hierarchy
of regions, fragile sites, virus integration sites, amplification 
hotspots, and cancer genes. We use banded matrices to visualize the
clusters from mixture models and the rules from semantic data mining
algorithm. The proposed method is suitable for both labeled and 
unlabeled data as cluster indices can be used as class labels 
in semantic data mining. Furthermore, banded matrix provides the 
visualization aspect to the analysis for detailed study of the data. 
Thus, the method is also suitable for rigorous analysis of 
multiresolution data.

Every system in the world is connected with one another and 
each system effects the other system. Consequently, understanding
one system can help understand another system better. In this 
scenario, knowledge or understanding  of one system can be used 
as a background information to understand another system. These
methods are applicable in bioinformatics as interacting systems 
produce different datasets. Similarly, the proposed methodology 
could be applicable in natural language
processing~\cite{manning1999foundations} 
because the additional background  knowledge in natural language 
processing are available in form of ontologies such as the 
semantic web.

The three part methodology proposed in~\citepub{j2} takes as an input only
data in a single resolution. Multiresolution analysis is achieved by using 
the taxonomy of multiresolution hierarchy as an additional background knowledge 
to the methodology. In the future, the semantic pattern mining algorithms
can be developed to include data in multiple resolutions simultaneously
in addition to the taxonomy of hierarchy of regions. 

%We have not 
%considered the class labels available for the data in coarse resolution.








