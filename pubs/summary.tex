\chapter[Summary and Conclusions]{Summary and Conclusions}
\markboth{Summary and Conclusions}{}
\label{ch:summary}

\begin{fquote}[Graham Greene]A story has no beginning or end: 
arbitrarily one chooses that moment of experience from which 
to look back or from which to look ahead.
\fqsource{The End of the Affair (1951)} \end{fquote}

\begin{synopsis}
This chapter summarizes the contributions of the thesis
and draws conclusion from the research. The chapter also 
discusses the overall future research perspectives in 
multiresolution modelling domain. 
%The specific improvements
%to the methods developed in this thesis is developed
%in Chapter~\ref{ch:discussion}.
\end{synopsis}

\section{Summary}
\label{s:finsum}



In traditional machine learning and data mining scenario 
data analysed is from a single source represented in 
a single resolution. In current age of big data, the 
challenge is to analyse massive set of datasets, i.e., the 
challenge is to analyse  multiple datasets within a single analysis. The multiple 
datasets can be available in different representations. 
Analysis of data in multiple representations needs
methods and algorithms suitable for different 
situations and application areas. Analysis of 
data in multiple representations within a single analysis
framework also caters the needs of data hungry algorithms.
%and helps avoid overfitting and underfitting.

The work in this 
thesis has concentrated in developing algorithms
and methods to address the challenges in 
modelling data in multiple representations. 
In this thesis, multiple representations aspect is 
provided by the data represented in multiple resolutions.
The algorithms especially covers mixture models and 
semantic data mining methods. Different methods and 
algorithms have been developed to 
analyse multiresolution data suitable for different 
situations and application areas. 

The data transformation methods proposed in the thesis transforms 
data across different resolutions to integrate datasets in different 
resolutions providing an opportunity to analyse data in a single 
resolution. Additionally, a computationally efficient algorithm to 
train a series of mixture models to aid model selection algorithms 
is developed in the thesis. Similarly, an algorithm based on merging 
of mixture components to model multiresolution data produces models 
in each resolution incorporating information from other data resolutions. 
In addition, a multiresolution mixture model uses the domain knowledge
to design multiresolution mixture components which are individually 
functional as Bayesian networks. Furthermore, a semantic data mining 
algorithm developed in this thesis uses knowledge of hierarchy
of multiresolution data and other background knowledge
to extract rules from the data. The algorithms and methods
provide plausible improvements in multiresolution data 
analysis compared to the individual analysis in the single 
resolution data.

\section{Future Work}
\label{s:finfuture}

The multiresolution analysis methodology developed in this thesis 
are at its initial stage. 
The thesis  forms the foundations for multiresolution modelling
and the algorithms and methods proposed in the thesis need further
research on the scope and general applicability. The methods
are tested only on datasets such as the chromosomal aberrations 
datasets, publicly available datasets, and simulated 
datasets. However, the methods have not been developed as a tool 
with rigorous testing for general applicability. The improvements 
necessary for each of the developed methods and algorithms 
are discussed in Chapter~\ref{ch:discussion}. This section discusses
the future improvements in overall multiresolution analysis domain. 
It includes developing the EM algorithm to learn 
the multiresolution components of the mixture models.
The EM algorithm used in this thesis learns the maximum 
likelihood parameters when networks were arranged as vectors.


Throughout the thesis, mixture models are used in hard clustering
setting, i.e., one sample is only associated with one component 
distribution generating the maximum posterior probability. 
Mixture models can also be used in a soft clustering
setting where posterior probability can be used to assign a 
sample to more than one component distribution. Soft clustering 
setting is beneficial in the chromosomewise analysis of 
chromosomal aberrations data because some cancer samples with 
the same known cancer labels can be grouped in two different
clusters. Soft clustering of chromosomal aberrations data
can also be justified because of the heterogeneous nature 
of cancer. 

In chromosomewise analysis, two exactly similar cancer samples
can be labelled as two different cancers because other chromosomes
that are likely to discriminate cancers will be ignored in the 
current analysis. Furthermore, we have 73 different types  of 
cancer labels for data in coarse resolution. Therefore, we can 
use multiclass classification to analyse the data. In a broader 
context, multiresolution multiclass classification can be a way
forward in analysis of multiresolution data. 

We need  to consider multiresolution  data because of the large  
number of cancer types and smaller number of samples making 
multiclass classification a challenging task.
Furthermore, labels are unavailable for data in fine resolution.
In such situations, learning from ambiguous labels~\cite{hullermeier2006} 
or partial labels~\cite{cour2011} using clustering labels or the 
cancer types can help in the analysis of chromosomal
aberrations data. Finally, analysis of multiresolution 
modelling also requires visualisation of the data as 
well as the results. Therefore, visualisation is also 
another direction for future work. In~\citepub{j2}, we use banded 
matrix to visualise rules and cluster only in single resolution. 
Initial ideas to visualise multiresolution can borrow from a 
popular visualisation method in information 
visualisation known as the Fish eye view~\cite{furnas86}.
Similar to multiresolution data, Fish eye view also visualises
data, providing users a detailed and also a global view.

%which is popular in information visualization.
%3. Large number of cancer cases and smaller number of samples .... 
%multiclass classification
%exactly similar sample can  be associated with different 
%cancers when data from other chromosomes are discarded.
%1. Mixture models in context of soft clustering and multiclass 
%classification for cases within class but different clusters

%4. Visualization of multiresolution data ... Fish eye view similarity 
%Improvements are necessary on the methods 
%before being used as tools in multiresolution modelling. 